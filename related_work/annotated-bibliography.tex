\documentclass[11pt]{article}
\makeatletter

\usepackage{comment} % enables the use of multi-line comments (\ifx \fi) 
\usepackage{lipsum} %This package just generates Lorem Ipsum filler text. 
\usepackage{graphicx}
\usepackage{fullpage} % changes the margin
\usepackage{xspace}

\newcommand{\ninep}{\textsc{9P}\xspace}
\newcommand{\amoeba}{\textsc{Amoeba}\xspace}
\newcommand{\locutus}{\textsc{Locutus}\xspace}
\newcommand{\plannine}{\textsc{Plan 9}\xspace}
\newcommand{\rosgig}{R-OSGi Deployment Tool\xspace}
\newcommand{\rosgi}{R-OSGi\xspace}
\newcommand{\cde}{\textsc{CDE}\xspace}

\begin{document}
The \rosgig, a plugin for the Eclipse IDE, builds on top of the \rosgi module system, which is a dynamically-loadable module for Java applications based on the OSGi standard~\cite{10.5555/1785080.1785082, 10.1145/1328279.1328290}.   \rosgig includes a graphical plugin for visualizing deployments as well as a deployment agent that coordinates distribution of software.  The deployment agent performs a static analysis to determine which OSGi bundles need to be deployed to which machines; users graphically interact with the tool to place modules on remote hosts.

\rosgig and \rosgi are limited to Java applications built using \rosgi bundles; futhermore, they are intended to facilitate graphical partitioning of an application to a distributed setting.

The \amoeba operating system was a UNIX-like operating system designed to completely erase the distinction between local and remote processes~\cite{10.1109/2.53354}.  However, programs must be written using the \amoeba object system in order to gain distributed capabilities, which severly limited its appeal.  The \plannine operating system, by contrast, approached distributed applications in a more lightweight fashion, by designing a form of distributed IPC modeled to appear like ordinary UNIX files, but were built on top of a distributed network protocol called \ninep~\cite{DBLP:journals/csys/PikePDFTT95}.  \plannine required modest rewrites of applications in the \plannine dialect of C.  Ironically, many of the requirements of \plannine C are now considered good programming practices in modern C.  Since \plannine, most operating systems have gained network capabilities, for example, the secure shell protocol~\cite{10.5555/1267569.1267573}, providing for a similar kind of file- and pipe-based IPC in ordinary operating systems.

\locutus has no inherent language limitations, and it is intended to automatically distribute an application wholesale to a remote host without user intervention or manual partitioning.

\bibliographystyle{acm}
\bibliography{refs}

The \cde package is a piece of software used for packing up Linux applications so that they can be run on other machines without having to specify the dependencies needed or copy over the entire system. ~\cite{10.1109/MCSE.2012.36} \cite{10.5555/2002181.2002202}. The design of the \cde package is similar to the original design for \locutus, where the application to be packed is executed on the local machine, system calls detect which files are accessed, and those files are packed up so that the entire bundle, when sent to another machine, can be used to execute the program. Furthermore, \cde has a mode of execution called the \textit{application streaming mode}, to run an application where the required files are fetched on-demand over SSH and cached locally on the user's machine.

\end{document}

