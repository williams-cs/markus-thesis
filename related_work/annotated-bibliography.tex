\documentclass[11pt]{article}
\makeatletter

\usepackage{comment} % enables the use of multi-line comments (\ifx \fi) 
\usepackage{lipsum} %This package just generates Lorem Ipsum filler text. 
\usepackage{graphicx}
\usepackage{fullpage} % changes the margin
\usepackage{xspace}

\newcommand{\locutus}{\textsc{Locutus}\xspace}
\newcommand{\rosgig}{R-OSGi Deployment Tool\xspace}
\newcommand{\rosgi}{R-OSGi\xspace}

\begin{document}
The \rosgig, a plugin for the Eclipse IDE, builds on top of the \rosgi module system, which is a dynamically-loadable module for Java applications based on the OSGi standard~\cite{10.5555/1785080.1785082, 10.1145/1328279.1328290}.   \rosgig includes a graphical plugin for visualizing deployments as well as a deployment agent that coordinates distribution of software.  The deployment agent performs a static analysis to determine which OSGi bundles need to be deployed to which machines; users graphically interact with the tool to place modules on remote hosts.

\rosgig and \rosgi are limited to Java applications built using \rosgi bundles; futhermore, they are intended to facilitate graphical partitioning of an application to a distributed setting.  By contrast, \locutus has no inherent language limitations, and it is intended to automatically distribute an application wholesale to a remote host without user intervention or manual partitioning.

\bibliographystyle{acm}
\bibliography{refs}

\end{document}

